\documentclass[11pt]{article}
\usepackage{geometry}
\geometry{letterpaper} 

\usepackage{framed}
\newenvironment{quotationb}%
{\begin{leftbar}\begin{quotation}}%
{\end{quotation}\end{leftbar}}

\usepackage{fancyvrb}
\CustomVerbatimCommand{\Source}{VerbatimInput}{frame=single, numbers=left}

\title{Scala for the Study of Programming Languages}
\author{Brian Howard}

\newcommand{\qhead}[1]{\vspace{5mm}\noindent\textbf{#1}\\}

\begin{document}
\maketitle
\section{Introduction}
This document provides an overview of the Scala language as it will
be used in our programming languages course. It assumes that you
are already familiar with programming in Java and/or C++; you should
already have learned some Scala or another functional language in
Foundations, but you are not expected to be an expert.

Before diving into the details, here is a description of Scala
written by Martin Odersky, the language's designer, for the main
Scala website, \texttt{http://www.scala-lang.org/}:
\begin{quotationb}\vspace{-5mm}
\qhead{A Scalable language}
Scala is an acronym for ``Scalable Language''. This means that Scala
grows with you. You can play with it by typing one-line expressions
and observing the results. But you can also rely on it for large
mission critical systems, as many companies, including Twitter,
LinkedIn, or Intel do.

To some, Scala feels like a scripting language. Its syntax is concise
and low ceremony; its types get out of the way because the compiler
can infer them. There's a REPL and IDE worksheets for quick feedback.
Developers like it so much that Scala won the ScriptBowl contest
at the 2012 JavaOne conference.

At the same time, Scala is the preferred workhorse language for
many mission critical server systems. The generated code is on a
par with Java's and its precise typing means that many problems are
caught at compile-time rather than after deployment.

At the root, the language's scalability is the result of a careful
integration of object-oriented and functional language concepts.

\qhead{Object-Oriented}
Scala is a pure-bred object-oriented language. Conceptually, every
value is an object and every operation is a method-call. The language
supports advanced component architectures through classes and traits.

Many traditional design patterns in other languages are already
natively supported. For instance, singletons are supported through
object definitions and visitors are supported through pattern
matching. Using implicit classes, Scala even allows you to add new
operations to existing classes, no matter whether they come from
Scala or Java!

\qhead{Functional}
Even though its syntax is fairly conventional, Scala is also a
full-blown functional language. It has everything you would expect,
including first-class functions, a library with efficient immutable
data structures, and a general preference of immutability over
mutation.

Unlike with many traditional functional languages, Scala allows a
gradual, easy migration to a more functional style. You can start
to use it as a ``Java without semicolons''. Over time, you can
progress to gradually eliminate mutable state in your applications,
phasing in safe functional composition patterns instead. As Scala
programmers we believe that this progression is often a good idea.
At the same time, Scala is not opinionated; you can use it with any
style you prefer.

\qhead{Seamless Java Interop}
Scala runs on the JVM. Java and Scala classes can be freely mixed,
no matter whether they reside in different projects or in the same.
They can even mutually refer to each other, the Scala compiler
contains a subset of a Java compiler to make sense of such recursive
dependencies.

Java libraries, frameworks and tools are all available. Build tools
like ant or maven, IDEs like Eclipse, IntelliJ, or Netbeans,
frameworks like Spring or Hibernate all work seamlessly with Scala.
Scala runs on all common JVMs and also on Android.

The Scala community is an important part of the Java ecosystem.
Popular Scala frameworks, including Akka, Finagle, and the Play web
framework include dual APIs for Java and Scala.

\qhead{Functions are Objects}
Scala's approach is to develop a small set of core constructs that
can be combined in flexible ways. This applies also to its
object-oriented and functional natures. Features from both sides
are unified to a degree where Functional and Object-oriented can
be seen as two sides of the same coin.

Some examples: Functions in Scala are objects. The function type
is just a regular class. The algebraic data types found in languages
such as Haskell, F$\sharp$ or ML are modelled in Scala as class
hierarchies. Pattern matching is possible over arbitrary classes.

\qhead{Future-Proof}
Scala particularly shines when it comes to scalable server software
that makes use of concurrent and synchronous processing, parallel
utilization of multiple cores, and distributed processing in the
cloud.

Its functional nature makes it easier to write safe and performant
multi-threaded code. There's typically less reliance on mutable
state and Scala's futures and actors provide powerful tools for
organizing concurrent system at a high-level of abstraction.

\qhead{Fun}
Maybe most important is that programming in Scala tends to be very
enjoyable. No boilerplate, rapid iteration, but at the same time
the safety of a strong static type system. As Graham Tackley from
the Guardian says: ``We've found that Scala has enabled us to deliver
things faster with less code. It's reinvigorated the team.''
\end{quotationb}

For our purposes, the main advantages of Scala, beyond its kinship
to Java, are its modern, multi-paradigm design, and its ``batteries
included'' library support for easily defining new languages and
language extensions. In addition to comparing Scala's language
features with those of other common real-world languages, we will
be exploring the basics of language implementation by constructing
interpreters for several ``toy'' languages.

The following sections will give a series of examples of Scala code
alongside equivalent code in C++, Standard ML, and Java. This is
not intended to give exhaustive coverage of the language, but rather
to gain enough familiarity with Scala that it may be used for
exercises and small projects.

\section{Scala as an Imperative Language}
Scala inherited much of its basic syntax from C, C++, and Java. The
two most noticeable departures stem from the Scala compiler doing
extra work to ``infer'' some pieces of syntax that don't add much
information:
\begin{itemize}
\item Semicolons are inferred at the end of a statement if it
coincides with the end of a line;
\item Types are inferred from the initial values for variable
declarations. One consequence of this is that it is more convenient
for types, when needed, to be annotated as \verb|thing: type|
(Pascal-style), instead of the C-style \verb|type thing|.
\end{itemize}

First, here are programs to provide a friendly greeting to each
name given on the command line:
\Source[label=C++: HelloAll.cpp]{src/main/cpp/HelloAll.cpp}
\Source[label=Java: HelloAllJ.java]{src/main/java/HelloAllJ.java}
\Source[label=Scala: HelloAllS.scala]{src/main/scala/HelloAllS.scala}

Those \texttt{while} loops in C++ and Java would more typically be
written as \texttt{for} loops. Scala has a more general form of
\texttt{for} loop that we will see more of later; the omission of
the C-style \texttt{for} loop (and the increment operator \verb|++|)
is meant to discourage programs that rely too much on side-effects.
\Source[label=Java: HelloAll2J.java]{src/main/java/HelloAll2J.java}
\Source[label=Scala: HelloAll2S.scala]{src/main/scala/HelloAll2S.scala}

The next set of examples show insertion sort, to demonstrate breaking
up a program into functions and passing parameters:
\Source[label=C++: ISort.cpp]{src/main/cpp/ISort.cpp}
\Source[label=Scala: ISort.scala]{src/main/scala/ISort.scala}

\section{Scala as a Functional Language}
We will cover a wide variety of language features that are typically
associated with functional languages, including:
\begin{itemize}
\item Emphasis on side-effect-free expressions and recursive functions;
\item Immutable data structures, often defined recursively;
\item Algebraic data types and pattern matching;
\item Higher-order functions, and anonymous function values.
\end{itemize}
The functional part of Scala's design was strongly influenced by
both Standard ML (SML) and Haskell. The design of Scala's actor
system, to be discussed when we look at concurrency, was based on
Erlang.

Here is an example comparing SML to Scala that shows most of the
above features. It implements a binary search tree plus some utility
functions.
\Source[label=SML: BSTDemo.sml]{src/main/sml/BSTDemo.sml}
\Source[label=Scala: BSTDemo.scala]{src/main/scala/BSTDemo.scala}

\section{Scala as an Object-Oriented Language}
Since it is hard to see the utility of an object-oriented approach
in a small program, here is a somewhat longer example, first in C++
and then in Scala, of defining classes to represent and use expression
trees. In this code, an expression (represented by the base class
\verb|Expr|) is either an integer constant (\verb|Const_Expr|) or
the sum of two subexpressions (\verb|Plus_Expr|); later we will
build on this example to handle more cases. The common interface
provided by the base class consists of two methods (besides the
virtual destructor needed in C++ to free up memory properly on
deletion): \verb|eval()| returns the integer result of evaluating
the expression, and \verb|accept(visitor)| uses the Visitor design
pattern to apply some operation to each node in the expression tree.
In the example, two subclasses of \verb|Visitor| are shown:
\verb|Infix_Visitor| and \verb|Postfix_Visitor| will print out the
expression in either infix or postfix form.
\Source[label=C++: ExprDemo.cpp]{src/main/cpp/ExprDemo.cpp}
\Source[label=Scala: ExprDemo.scala]{src/main/scala/ExprDemo.scala}

Scala improves on the ``boilerplate'' (repetitive code that is
copied with only minor changes for each use) required by C++ in
several ways:
\begin{itemize}
\item Simple constructors, private data members, and (where desired)
getters are combined in class parameters (if the parameter is
prefixed with \verb|val|, it will have a getter; if the prefix is
\verb|var|, then it will also have a setter);
\item Destructors are not needed, because Scala (and Java) use
garbage collection to manage memory automatically.
\end{itemize}
Furthermore, as noted in the quote from Odersky, common design
patterns such as Visitor are not needed when Scala's functional
features (in this case, pattern matching) are available. Here is a
rewrite of the above code in a more idiomatic Scala style:
\Source[label=Scala: ExprDemo2.scala]{src/main/scala/ExprDemo2.scala}

\end{document}

